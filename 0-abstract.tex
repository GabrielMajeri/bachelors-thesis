\begin{abstractpage}

\begin{abstract}{romanian}
Primul pas în îmbunătățirea performanței programelor este determinarea porțiunilor de cod care consumă cel mai mult timp de execuție. În acest scop se pot folosi profilatoare tracing, care dau rezultate exacte cu un cost suplimentar de analiză mare, sau profilatoare statistice, care nu afectează timpul de execuție dar au o acuratețe mult mai scăzută.

În această lucrare introduc un profilator de performanță hibrid, care îmbină adaptiv metodele de mai sus pentru a obține rezultate precise cu un cost adițional neglijabil. De asemenea, poate fi configurat să măsoare și alți indicatori ai performanței, cum ar fi numărul de rateuri de cache.
\end{abstract}

\begin{abstract}{english}
The first step in the improvement of program performance is determining the parts of the code which consume the most of the execution time. For this purpose tracing profilers can be used, which provide exact results with a large overhead, or statistical profilers, with little impact on execution time but lower accuracy.

In this paper I introduce a hybrid performance profiler, which adaptively combines the methods described above to obtain precise results with negligible overhead. Furthermore, it can be configured to measure other performance indicators, such as the number of cache misses.
\end{abstract}

\end{abstractpage}